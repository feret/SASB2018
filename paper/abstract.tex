%!TeX spellcheck = en-GB
Rule-based languages, such as Kappa and BNGL, allow for the description of very combinatorial models of interactions between proteins. A huge (when not infinite) number of different kinds of bio-molecular compounds may arise
 due to proteins with multiple binding and phosphorylation sites. Knowing beforehand whether a model may involve an infinite number of different kinds of bio-molecular compounds is crucial for the modeller. On the first hand, having an infinite number of kinds of bio-molecular compounds is sometimes a hint for modelling flaws: forgetting to specify
the conflicts among binding rules is a common mistake. On the second hand,
it impacts the choice of  the semantics for the models (among stochastic, differential, hybrid).

In this paper, we introduce a data-structure to abstract the potential unbounded polymers that may be formed in a rule-based model. This data-structure is a graph, the nodes and the edges of which are labelled with patterns. By construction,  every potentially unbounded polymer is associated to at least one cycle in that graph. This data-structure has two main advantages. Firstly, as opposed to site-graphs, one can reason about cycles without enumerating them (by the means of Tarjan's algorithm for detecting strongly connected components). Secondly, this data-structures may be combined easily with information coming from additional reachability analysis: the edges that are labelled with an overlap that is proved unreachable in the model may be safely discarded.
