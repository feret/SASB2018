%!TeX spellcheck = en-GB
\documentclass[landscape,20pt]{transparents2e}

\usepackage{concrete}
\usepackage{conf}
\usepackage[latin1]{inputenc}
\usepackage[T1]{fontenc}
\usepackage{galois}
\usepackage{url}
\usepackage{theorem}
\usepackage{float}
\usepackage{oldgerm}
\usepackage{xspace}
\usepackage{epic}
\usepackage{ae}
\usepackage{eepic}
\usepackage{ulem}

\usepackage{myfigures}
\usepackage{titlesec}
\usepackage{rotating}
\usepackage{epsfig}
\usepackage[dvipsnames]{color}
\usepackage{NamedColors}
\usepackage[psamsfonts]{amsfonts}
\usepackage{amsfonts,amssymb,amsmath,stmaryrd}
%\usepackage{ecltree}
\usepackage{array}
\usepackage[color,matrix,curve,graph,frame,all]{xy}
\usepackage{fancybox}
\usepackage{picinpar}
\usepackage{pdfpages}
\usepackage{stmaryrd}
\usepackage{lscape,graphicx}
\usepackage{amsmath,amscd}
\usepackage{multirow}
\usepackage{boolean}
\usepackage{selection}
%\usepackage{ulem}
\usepackage{concrete}
\newcommand{\rgives}{$\longleftrightarrow$}
\newcommand{\gives}{$\longrightarrow$}
\newcommand{\conc}[1]{\text{$\left[#1\right]$}}
\newcommand{\myfbox}[1]{\scalebox{0.7}{\fbox{#1}}}
\newcommand{\ldot}{\text{.}}
\definecolor{green1}{rgb}{0,0.4,0}
\definecolor{red2}{rgb}{0.75,0.1,0.1}
\definecolor{blue1}{rgb}{0,0,0.5}
\definecolor{purple}{rgb}{0.5,0.1,0.5}
\definecolor{grey}{rgb}{0.5,0.5,0.5}
\definecolor{white}{rgb}{1,1,1}
\newcommand{\white}{\textcolor{white}}
\newcommand{\grey}{\textcolor{grey}}
\newcommand{\red}{\textcolor{red}}
\newcommand{\blue}{\textcolor{BlueViolet}}
\newcommand{\black}{\textcolor{Black}}
\newcommand{\green}{\textcolor{green1}}
\newcommand{\keep}[2]{#1{#2}{}}
%\newcommand{-}{-}
\newcommand{\fun}{\textcolor{purple}}

\normalem


\newcommand{\myfrac}[2]{\frac{\strut \displaystyle #1}{\strut \displaystyle #2}}
\renewcommand{\frametitle}[1]{\red{\HUGE #1 }}
\newcommand{\articles}[1]{\green{#1}}

\newcommand{\coloredpart}[5]{\ifthenelse{\value{#2}=#3}{#1{#4}#5}{#4}}%
\newcommand{\redpart}{\coloredpart{\red}}
\newcommand{\comment}[1]{}

\newcommand{\displayplan}[2]{\begin{slide}{\frametitle{On the menu today}}

\huge

\vfill

{#1}
\setcounter{mainpart}{0}\begin{enumerate}
\item \addtocounter{mainpart}{1}\redpart{superpartie}{\value{mainpart}}{Rule-based modelling}{}%
\item \addtocounter{mainpart}{1}\redpart{superpartie}{\value{mainpart}}{Kappa}{}%
\item \addtocounter{mainpart}{1}\redpart{superpartie}{\value{mainpart}}{Unbounded bio-molecular compounds}{}%
\item \addtocounter{mainpart}{1}\redpart{superpartie}{\value{mainpart}}{The graph of the sites}{}
\item \addtocounter{mainpart}{1}\redpart{superpartie}{\value{mainpart}}{The graph of the edges}{}%
\item \addtocounter{mainpart}{1}\redpart{superpartie}{\value{mainpart}}{Refinement}{}%
\item \addtocounter{mainpart}{1}\redpart{superpartie}{\value{mainpart}}{Conclusion}{}%
\end{enumerate}
{#2}

\vfill

\end{slide}}


\newcounter{superpartie}
\setcounter{superpartie}{0}
\newcommand{\superplan}{\displayplan{\addtocounter{superpartie}{1}}{}}

\begin{document}

\begin{slide}{}
\thispagestyle{empty}
\newcommand{\Hrule}{\rule{\linewidth}{1mm}}

\vspace*{-2.1cm}

\begin{center}
{\bf \Huge{\confname}} \bigskip\\
\end{center}

\vspace*{0.2cm}

\begin{center}
{\bf{\red{\HUGE{Proving the absence of unbounded polymers
in rule-based models}}}} \bigskip\\
\end{center}

\vspace*{0.2cm}

\begin{center}
{\blue {{\Huge \bf Jerome Feret}}}  \smallskip\\
\huge{DI - \'ENS \bigskip}
%\huge{\lang{\'Equipe Antique}{Team Antique}} \bigskip\\
\end{center}

\vspace*{0.2cm}

\begin{center}
\begin{minipage}{\linewidth}
\hfill\hfill
\begin{minipage}{0.08\linewidth}
\includegraphics[height=1.5cm]{inr_logo_cherch_UK_coul.png}
\end{minipage}\hfill
\begin{minipage}{0.08\linewidth}
  \includegraphics[height=2.92cm]{ENS_Logo.png}
\end{minipage}\hfill
\begin{minipage}{0.08\linewidth}
\includegraphics[height=2.5cm]{CNRSfilaire-grand.jpg}
\end{minipage}\hfill
\begin{minipage}{0.1\linewidth}
  \includegraphics[height=1.81cm]{UPL3732886268454158059_logoPSLstar_RU_rvb.png}
\end{minipage}\hfill\hfill\mbox{}
\end{minipage}
\end{center}


\vspace*{0.4cm}

\begin{center}
\red{\url{http://www.di.ens.fr/~feret}}
\end{center}


\vspace*{0.4cm}

\begin{center}
\huge{Joint work with Pierre Boutillier and Aurelie Faure de Pebeyre}
\end{center}

\vspace*{0.4cm}


\begin{center}
\huge{\ladate}
\end{center}

\vfill

\end{slide}

%\comment

{\superplan}


\begin{slide}{\frametitle{Signalling Pathways}}
\begin{center}
\hspace*{2cm}\scalebox{0.8}{\begin{minipage}{\linewidth}\includegraphics[height=450pt,width=545pt]{generated_pictures/EGFR_signaling_pathway.png}\end{minipage}}
\end{center}
\hfill \articles{Eikuch, 2007}

\end{slide}

\begin{slide}{\frametitle{Bridging the gap between\ldots}}

\vfill

\begin{minipage}{\linewidth}
\begin{minipage}{0.35\linewidth}
\scalebox{0.4}{\begin{minipage}{\linewidth}\includegraphics{generated_pictures/kitano.pdf}\end{minipage}}
%\articles{Oda, Matsuoka, Funahashi, Kitano, Molecular Systems Biology, 2005}
\end{minipage}
\begin{minipage}{0.2\linewidth}
\mbox{}
\end{minipage}
\begin{minipage}{0.35\linewidth}
\scalebox{0.65}{$\left\{\begin{array}{l}
\frac{d x_1}{dt} = -k_1\cdot x_1\cdot x_2 + k_{-1}\cdot x_3\smallskip\cr
\frac{d x_2}{dt} = -k_1\cdot x_1\cdot x_2 + k_{-1}\cdot x_3\smallskip\cr
\frac{d x_3}{dt} = k_1\cdot x_1\cdot x_2 - k_{-1}\cdot x_3 + 2\cdot k_2\cdot x_3\cdot x_3 - k_{-2}\cdot x_4\smallskip\cr
\frac{d x_4}{dt} = k_2\cdot x_3^2 -k_2 \cdot x_4 + \frac{v_4\cdot x_5}{p_4+x_5} - k_3\cdot x_4 - k_{-3}\cdot x_5\smallskip\cr
\frac{d x_5}{dt} = \cdots \smallskip\cr
\hspace*{1cm}\vdots\smallskip\cr
\frac{d x_n}{dt} = -k_1\cdot x_1\cdot c_2 + k_{-1}\cdot x_3\smallskip\cr
\end{array}\right.$}
\end{minipage}
\end{minipage}


\vfill

\noindent\begin{minipage}{1\linewidth}
\begin{minipage}{0.35\linewidth}
\begin{center}
\Huge{\red{\textbf{knowledge representation}}}
\end{center}
\end{minipage}
\begin{minipage}{0.2\linewidth}
\begin{center}
\Huge{\red{\textbf{and}}}
\end{center}
\end{minipage}
\begin{minipage}{0.40\linewidth}
\begin{center}
  \Huge{\red{\textbf{models of the behaviour of systems}}}
  \end{center}\end{minipage}\end{minipage}


\vfill\mbox{}
\end{slide}

\keep{\rulebasedmodelinintro}{
\begin{slide}{\frametitle{\lang{Réécriture de graphes à sites}{Site-graphs rewriting}}}

\begin{center}
\scalebox{0.7}{\includegraphics{generated_pictures/dimerisation.pdf}}
\end{center}




\vfill

\hfill\begin{minipage}{0.8\linewidth}
{\huge \begin{itemize}
\item \lang{un langage proche des cartes d'interaction~;}{a language close to knowledge representation;}
\item \lang{des règles faciles à modifier~;}{rules are easy to update;}
\item \lang{une représentation compacte}{a compact description of models}.
\end{itemize}}
\end{minipage}\hfill\mbox{}

\vfill

\end{slide}}

\begin{slide}{\frametitle{\lang{Choix de sémantiques}{Choices of semantics}}}


\begin{center}
\blue{\fbox{\scalebox{0.25}{\includegraphics{generated_pictures/dimerisation.pdf}}}}
\end{center}
\hspace*{3.cm}\begin{minipage}{0.65\linewidth}%
\scalebox{0.3}{%
\begin{minipage}{\linewidth}%
\input{generated_pictures/compilation.pdf_t}%
\end{minipage}}
\end{minipage}%

\hspace*{-1cm}\blue{\fbox{\black{\begin{minipage}{0.19\linewidth}
\scalebox{0.35}{\begin{minipage}{\linewidth}
\includegraphics{generated_pictures/sos_contact_map.pdf}
\end{minipage}}
\begin{center}{\lang{carte d'interaction}{interaction map}}\bigskip\\\end{center}
\end{minipage}
}}}\hspace*{2cm}
\blue{%
\fbox{%
\black{%
\begin{minipage}{0.32\linewidth}%
\scalebox{0.43}{\includegraphics{generated_pictures/pic1x.pdf}}
\begin{center}
\lang{chaîne de Markov}{Markov chain}\bigskip\\
\end{center}%
\end{minipage}%
}}}%
\hspace*{1.5cm}
\blue{\fbox{\black{\begin{minipage}{0.4\linewidth}
\mbox{}\smallskip\\\hspace*{5mm}\scalebox{0.55}{$\begin{cases}
\frac{d x_1}{dt} = -k_1\cdot x_1\cdot x_2 + k_{-1}\cdot x_3\smallskip\cr
\frac{d x_2}{dt} = -k_1\cdot x_1\cdot x_2 + k_{-1}\cdot x_3\smallskip\cr
\frac{d x_3}{dt} = k_1\cdot x_1\cdot x_2 - k_{-1}\cdot x_3 + 2\cdot k_2\cdot x_3\cdot x_3 - k_{-2}\cdot x_4\smallskip\cr
\frac{d x_4}{dt} = k_2\cdot x_3^2 -k_2 \cdot x_4 + \frac{v_4\cdot x_5}{p_4+x_5} - k_3\cdot x_4 - k_{-3}\cdot x_5\smallskip\cr
\frac{d x_5}{dt} = \cdots \smallskip\cr
\hspace*{1cm}\vdots\smallskip\cr
\frac{d x_n}{dt} = -k_1\cdot x_1\cdot c_2 + k_{-1}\cdot x_3\smallskip\cr
\end{cases}$} \begin{center}\lang{équations différentielles}{ordinary differential equations}\bigskip\\\end{center} \end{minipage}}}}

%\end{center}
\end{slide}



\begin{slide}{\frametitle{Complexity walls}}

\vspace*{-19.5cm}
\scalebox{1.2}
{\begin{minipage}{\linewidth}
\includegraphics{generated_pictures/walls.pdf}
\end{minipage}}
\end{slide}


\begin{slide}{\frametitle{\lang{Les abstractions offrent différentes perspectives sur les modèles}{Abstractions offer different perspectives on models}}}

\begin{minipage}{0.55\linewidth}
\hspace*{8mm}\scalebox{0.6}{\includegraphics{generated_pictures/sos.pdf}}

\hspace*{2cm}{\blue{\lang{sémantique concrète}{concrete semantics}}}
\end{minipage}
\begin{minipage}{0.44\linewidth}
\hspace*{-5mm}\scalebox{0.3}{\includegraphics{generated_pictures/pastedGraphicbis.pdf}}
\hspace*{1cm}
\scalebox{0.3}{\includegraphics{generated_pictures/pastedGraphic2bis.pdf}}


\hspace*{1cm}{\blue{\lang{traces causales}{causal traces}}}

\end{minipage}

\vspace*{1cm}

\begin{minipage}{0.49\linewidth}
\hspace*{2.cm}\scalebox{0.4}{\includegraphics{generated_pictures/contact_map_annotated.pdf}}

\hspace*{2cm}{\blue{\lang{flot de l'information}{information flow}}}
\end{minipage}
\begin{minipage}{0.5\linewidth}
\scalebox{0.6}{\includegraphics{generated_pictures/sos_ode_fragments.pdf}}

\hspace*{2cm}{\blue{\lang{projection exacte}{exact projection}}}

\hspace*{0.6cm}{\blue{\lang{de la sémantique différentielle}{\hspace*{5mm}of the ODE semantics}}}
\end{minipage}
\end{slide}


{\superplan}


\begin{slide}{\frametitle{Bio-molecular compound}}
\begin{minipage}{\linewidth}
\begin{center}\scalebox{0.9}{\includegraphics{generated_pictures/species.pdf}}
\end{center}
\vfill
\end{minipage}
\end{slide}

\begin{slide}{\frametitle{Receptor activation}}
\begin{minipage}{\linewidth}
\begin{center}
\scalebox{1.2}{\includegraphics{generated_pictures/rule1.pdf}}
\end{center}
\end{minipage}
\end{slide}

\begin{slide}{\frametitle{Asymmetric dimerisation}}
\begin{minipage}{\linewidth}
\begin{center}
\scalebox{1.2}{\includegraphics{generated_pictures/rule11.pdf}}
\end{center}
\vspace*{1cm}
\begin{center}
\scalebox{1.2}{\includegraphics{generated_pictures/rule12.pdf}}
\end{center}
\end{minipage}
\end{slide}


\begin{slide}{\frametitle{Sequential unbinding}}
\begin{minipage}{\linewidth}
\begin{center}
\scalebox{1.2}{\includegraphics{generated_pictures/ruled11.pdf}}
\end{center}
\begin{center}
\scalebox{1.2}{\includegraphics{generated_pictures/ruled12.pdf}}
\end{center}
\begin{center}
\scalebox{1.2}{\includegraphics{generated_pictures/ruled13.pdf}}
\end{center}
\end{minipage}
\end{slide}


\begin{slide}{\frametitle{Phosphorylation}}

\vspace*{1cm}


\begin{minipage}{\linewidth}
\begin{center}
\scalebox{1.2}{\includegraphics{generated_pictures/rule2.pdf}}
\end{center}
\end{minipage}
\end{slide}

%\short
{
\begin{slide}{\frametitle{Don't care, Don't write}}
\begin{minipage}{\linewidth}
\begin{center}
\scalebox{1.2}{\includegraphics{generated_pictures/rule3.pdf}}
\end{center}
\end{minipage}

\vspace*{1cm}

\begin{equation*}
{\HUGE \neq}
\end{equation*}

\vspace*{1cm}

\begin{minipage}{\linewidth}
\begin{center}
\scalebox{1.2}{\includegraphics{generated_pictures/rule4.pdf}}
\end{center}
\end{minipage}
\end{slide}



\begin{slide}{\frametitle{Sequential unbinding \textcolor{blue}{(by side effects)}}}

  \begin{minipage}{\linewidth}
  \begin{center}
  \scalebox{1.2}{\includegraphics{generated_pictures/ruledse11.pdf}}
  \end{center}
  \begin{center}
  \scalebox{1.2}{\includegraphics{generated_pictures/ruledse12.pdf}}
  \end{center}
  \begin{center}
  \scalebox{1.2}{\includegraphics{generated_pictures/ruledse13.pdf}}
  \end{center}
  \end{minipage}
  \end{slide}

\begin{slide}{\frametitle{Creation/Suppression}}
\begin{minipage}{\linewidth}
\hspace*{10cm}\scalebox{1.2}{\includegraphics{generated_pictures/rule6.pdf}}\hspace*{2cm}\mbox{}
\end{minipage}

\vfill

\begin{minipage}{\linewidth}
\hspace*{6cm}\scalebox{1.2}{\includegraphics{generated_pictures/rule7.pdf}}
\end{minipage}
\end{slide}}


{\superplan}

\begin{slide}{\frametitle{Motivation}}

\vfill

\textcolor{red}{Potentially unbounded polymers}:

\vfill

\begin{enumerate}
   \item may naturally emerge in model with \textcolor{red}{self-assembling} of giant macro-molecules

   \begin{itemize}
     \item \textcolor{blue}{DNA} \textcolor{OliveGreen}{(Jean Krivine)},
     \item \textcolor{blue}{filaments} \textcolor{OliveGreen}{(Nathalie Theret)}, \item \textcolor{blue}{signalosome} \textcolor{OliveGreen}{(Hector Medica Abarca)};\end{itemize}

\vfill

   \item may result of \textcolor{red}{forgotten conflicts} between bonds;

   often the case when the model is extracted from a more abstract description
   \begin{itemize}
     \item \textcolor{blue}{Cell-Designer} \textcolor{OliveGreen}{(Luca Grieco)},
     \item \textcolor{blue}{natural language processing}
     \textcolor{OliveGreen}{(Sorger Lab)}
\end{itemize}


\end{enumerate}

\vfill

\end{slide}


\begin{slide}{\frametitle{Our goal}}


We want to prove the absence of unbounded polymers:\bigskip\\

\begin{minipage}{\linewidth}
\scalebox{0.8}{\includegraphics{generated_pictures/trimera.pdf}}
\begin{minipage}{1cm}\vspace*{-2.4cm}$\ldots$\end{minipage}
\scalebox{0.8}{\includegraphics{generated_pictures/trimerb.pdf}}
\end{minipage}

\end{slide}


\begin{slide}{\frametitle{Strengthened goal}}

  We want to prove the absence of unbounded polymers:\bigskip\\

  \begin{minipage}{\linewidth}
  \scalebox{0.8}{\includegraphics{generated_pictures/trimera.pdf}}
  \begin{minipage}{1cm}\vspace*{-2.4cm}$\ldots$\end{minipage}
  \scalebox{0.8}{\includegraphics{generated_pictures/trimerb.pdf}}
  \end{minipage}

\vspace*{1cm}

Instead, we will prove the absence of repeatable patterns:\bigskip\\

\begin{minipage}{\linewidth}
\begin{center}\scalebox{0.8}{\includegraphics{generated_pictures/repeatable_dimers.pdf}}
\end{center}
\end{minipage}

\end{slide}

\begin{slide}{\frametitle{Interaction map}}

  {\begin{minipage}{\linewidth}
  \begin{center}
    \scalebox{1.2}{\includegraphics{generated_pictures/contact_map.pdf}}
  \end{center}
  \end{minipage}}

\end{slide}

\begin{slide}{\frametitle{Interpretation}}

  {\begin{minipage}{\linewidth}
  \begin{center}
    \scalebox{0.7}{\includegraphics{generated_pictures/egfr_embed.pdf}}
  \end{center}
  \end{minipage}}

\end{slide}

\begin{slide}{\frametitle{Interaction map}}

  {\begin{minipage}{\linewidth}
  \begin{center}
    \scalebox{1.2}{\includegraphics{generated_pictures/contact_map.pdf}}
  \end{center}
  \end{minipage}}

\end{slide}

\begin{slide}{\frametitle{Conflicts}}

  {\begin{minipage}{\linewidth}
  \begin{center}
    \scalebox{1.2}{\includegraphics{generated_pictures/conflict_contact_map.pdf}}

    Interaction map.
  \end{center}
  \end{minipage}}

\end{slide}

\begin{slide}{\frametitle{Conflicts:}}

  \begin{minipage}{0.4\linewidth}
  \begin{center}
    \scalebox{0.8}{\includegraphics{generated_pictures/conflict_contact_map.pdf}}

    Interaction map.
  \end{center}
  \end{minipage}
  \begin{minipage}{0.55\linewidth}
    \label{fig:conflict:pattern}
  \scalebox{0.6}{\includegraphics{generated_pictures/conflict_r.pdf}}%
\hspace*{7mm}\scalebox{0.6}{\includegraphics{generated_pictures/conflict_s.pdf}}%
\hspace*{7mm}\scalebox{0.6}{\includegraphics{generated_pictures/conflict_g.pdf}}%
\hspace*{7mm}\scalebox{0.6}{\includegraphics{generated_pictures/conflict_rg.pdf}}%

\vspace*{0.6cm}

\scalebox{0.6}{\includegraphics{generated_pictures/conflict_grs.pdf}}%
\hspace*{7mm}\scalebox{0.6}{\hspace*{2mm}\includegraphics{generated_pictures/conflict_ggrs.pdf}}%
\hspace*{7mm}\scalebox{0.6}{\includegraphics{generated_pictures/conflict_rs.pdf}}%

\vspace*{0.6cm}

\hspace*{23mm}\scalebox{0.6}{\includegraphics{generated_pictures/conflict_gbisrs.pdf}}%
\hspace*{7mm}\scalebox{0.6}{\includegraphics{generated_pictures/conflict_gs.pdf}}%

\vspace*{0.6cm}

{\hfill}Bio-molecular compounds.{\hfill}\mbox{}
\end{minipage}

\end{slide}

%\begin{slide}{\frametitle{Alternative encoding}}

%\end{slide}

%\begin{slide}{\frametitle{Alternative encoding}}

%\end{slide}

%\begin{slide}{\frametitle{Alternative encoding}}

%\end{slide}

\begin{slide}{\frametitle{Self-loops}}

\vspace*{2cm}

\begin{minipage}{\linewidth}
  {\begin{minipage}{\linewidth}
  \begin{center}
    \scalebox{1.2}{\includegraphics{generated_pictures/self_contact_map.pdf}}

Interaction map.
  \end{center}
  \end{minipage}}
\end{minipage}

\end{slide}

\begin{slide}{\frametitle{Self-loops}}

\vspace*{2cm}

\begin{minipage}{\linewidth}
  \begin{minipage}{0.35\linewidth}
    {\begin{minipage}{\linewidth}
    \begin{center}
      \scalebox{1.2}{\includegraphics{generated_pictures/self_contact_map.pdf}}

Interaction map.
    \end{center}
    \end{minipage}}
  \end{minipage}
  \begin{minipage}{0.64\linewidth}
    \begin{center}
      \hspace*{5mm}\scalebox{1.2}{\includegraphics{generated_pictures/self_monomer.pdf}}
      \hspace*{5mm}\scalebox{1.2}{\includegraphics{generated_pictures/self_dimer.pdf}}

    Bio-molecular compounds.
  \end{center}

  \end{minipage}
\end{minipage}

\end{slide}



\begin{slide}{\frametitle{Several self-loops}}

\vspace*{2cm}

\begin{minipage}{\linewidth}
  {\begin{minipage}{\linewidth}
  \begin{center}
    \scalebox{1.2}{\includegraphics{generated_pictures/twoself_contact_map.pdf}}

Interaction map.
  \end{center}
  \end{minipage}}
\end{minipage}

\end{slide}

\begin{slide}{\frametitle{Several self-loops}}

\vspace*{2cm}

\begin{minipage}{\linewidth}
  \begin{minipage}{0.35\linewidth}
    {\begin{minipage}{\linewidth}
    \begin{center}
      \scalebox{1.2}{\includegraphics{generated_pictures/self_contact_map.pdf}}

Interaction map.
    \end{center}
    \end{minipage}}
  \end{minipage}
  \begin{minipage}{0.64\linewidth}
    \begin{center}
      \scalebox{1.2}{\includegraphics{generated_pictures/twoself_pattern.pdf}}

    A repeatable pattern.
  \end{center}

  \end{minipage}
\end{minipage}

\end{slide}

\newcommand{\core}[1]{%
\begin{minipage}{0.3\linewidth}
  \begin{center}
  \scalebox{0.8}{\includegraphics{generated_pictures/invariant_cm.pdf}}
Interaction map.
  \end{center}
\end{minipage}
  \begin{minipage}{0.69\linewidth}
  \begin{minipage}{0.55\linewidth}
     \begin{center}
    \scalebox{0.4}{\includegraphics{generated_pictures/rule1.pdf}}
    \end{center}
    \vspace*{0.2cm}
  \begin{center}
  \scalebox{0.4}{\includegraphics{generated_pictures/rule11.pdf}}
  \end{center}
  \vspace*{0.2cm}
  \begin{center}
  \scalebox{0.4}{\includegraphics{generated_pictures/rule12.pdf}}
\end{center}
\end{minipage}
\begin{minipage}{0.44\linewidth}
\begin{center}
\scalebox{0.4}{\includegraphics{generated_pictures/ruled11.pdf}}
\end{center}
  \vspace*{0.2cm}
\begin{center}
\scalebox{0.4}{\includegraphics{generated_pictures/#1.pdf}}
\end{center}
  \vspace*{0.2cm}
\begin{center}
\scalebox{0.4}{\includegraphics{generated_pictures/ruled13.pdf}}
\end{center}
\end{minipage}

  \begin{center}
  Rules.
  \end{center}
  \end{minipage}}

\begin{slide}{\frametitle{Invariants}}

\core{ruled12}


\end{slide}


\begin{slide}{\frametitle{Invariants}}

\core{ruled12}

\vspace*{1cm}

\begin{center}
  The repeatable pattern:

\scalebox{0.4}{\includegraphics{generated_pictures/invariant_pattern.pdf}}

  is not reachable.
\end{center}
\end{slide}

\begin{slide}{\frametitle{Breaking invariants}}

\core{rulewd}

\vspace*{1cm}

\begin{center}
The invariant does not hold any longer.\bigskip\\

The repeatable pattern:

\scalebox{0.4}{\includegraphics{generated_pictures/invariant_pattern.pdf}}

is now reachable.
\end{center}

\end{slide}

\newcommand{\coretriangle}{%
\begin{minipage}{0.4\linewidth}
  \begin{center}
  \scalebox{0.8}{\includegraphics{generated_pictures/abc_contact_map.pdf}}
Interaction map.
  \end{center}
\end{minipage}
  \begin{minipage}{0.59\linewidth}
  \begin{minipage}{\linewidth}
     \begin{center}
    \scalebox{1.}{\includegraphics{generated_pictures/bind_a_b.pdf}}
    \end{center}
    \vspace*{0.2cm}
  \begin{center}
  \scalebox{1.}{\includegraphics{generated_pictures/bind_b_c.pdf}}
  \end{center}
  \vspace*{0.2cm}
  \begin{center}
  \scalebox{1.}{\includegraphics{generated_pictures/bind_c_a.pdf}}
\end{center}
\end{minipage}

  \begin{center}
  Rules.
  \end{center}
  \end{minipage}}


\begin{slide}{\frametitle{Triangle}}

\coretriangle

\end{slide}

\begin{slide}{\frametitle{Triangle}}

  \begin{minipage}{0.55\linewidth}
  \begin{minipage}{\linewidth}
     \begin{center}
    \scalebox{1.}{\includegraphics{generated_pictures/bind_a_b.pdf}}
    \end{center}
    \vspace*{1cm}
  \begin{center}
  \scalebox{1.}{\includegraphics{generated_pictures/bind_b_c.pdf}}
  \end{center}
  \vspace*{1cm}
  \begin{center}
  \scalebox{1.}{\includegraphics{generated_pictures/bind_c_a.pdf}}
  \end{center}
  \end{minipage}

  \begin{center}
  Rules.
  \end{center}
  \end{minipage}
  \begin{minipage}{0.44\linewidth}
\includegraphics{generated_pictures/abc_a.pdf}
\hspace*{1cm}
\includegraphics{generated_pictures/abc_b.pdf}
\hspace*{1cm}
\includegraphics{generated_pictures/abc_c.pdf}

\vspace*{1cm}

\hspace*{1cm}\begin{minipage}{0.5\linewidth}
  \includegraphics{generated_pictures/abc_ab.pdf}

\vspace*{5mm}

\includegraphics{generated_pictures/abc_bc.pdf}

\vspace*{5mm}

\includegraphics{generated_pictures/abc_ca.pdf}
\end{minipage}

\vspace*{1cm}

\includegraphics{generated_pictures/abc_open_triangle.pdf}
\includegraphics{generated_pictures/abc_triangle.pdf}


    \begin{center}
      Bio-molecular compounds.
    \end{center}
  \end{minipage}

\end{slide}


\begin{slide}{\frametitle{Combinatorial complexity}}

\vfill

\begin{center}
  \includegraphics{generated_pictures/clique.pdf}
\end{center}

\vfill

\hfill On many model, elementary cycles are too numerous to enumerate them. \hfill\mbox{}

\end{slide}

{\superplan}

\begin{slide}{\frametitle{Repeatable patterns}}

\vfill

We want to prove the abscence of repeatable of the form:

\vfill

\begin{minipage}{\linewidth}
\begin{center}\scalebox{0.8}{\includegraphics{generated_pictures/repeatable_dimers.pdf}}
\end{center}
\end{minipage}

\vfill

\end{slide}

\begin{slide}{\frametitle{Transitions between sites}}

\vfill

We convert repeatable patterns into sequences of sites, by taking each other one: \bigskip\\


\begin{minipage}{\linewidth}
\begin{center}\scalebox{0.8}{\includegraphics{generated_pictures/repeatable_dimers.pdf}}
\end{center}
\end{minipage}

\vfill

\begin{enumerate}
  \item From left to right:

\hspace*{5cm}\includegraphics{generated_pictures/egfr_r.pdf}
\begin{minipage}{1cm}\vspace*{-2cm}$\stackrel{\begin{minipage}{1cm}\scalebox{0.3}{\includegraphics{generated_pictures/egfr_r.pdf}}\end{minipage}}{\longrightarrow}$\end{minipage} \includegraphics{generated_pictures/egfr_c.pdf}

\vfill

  \item From right to left:

  \hspace*{5cm}\includegraphics{generated_pictures/egfr_n.pdf}
  \begin{minipage}{1cm}\vspace*{-2cm}$\stackrel{\begin{minipage}{1cm}\scalebox{0.3}{\includegraphics{generated_pictures/egfr_c.pdf}}\end{minipage}}{\longrightarrow}$\end{minipage} \includegraphics{generated_pictures/egfr_r.pdf}

\end{enumerate}
\end{slide}

\begin{slide}{\frametitle{The graph of the sites \\\textcolor{blue}{EGF example}} }

  \scalebox{1.5}{\begin{minipage}{0.59\linewidth}\label{fig:abc:gs:gs}
  \xymatrix@C=2.cm@R=0.5cm{
  &&&\begin{minipage}{1.cm}
  \scalebox{0.35}{\includegraphics{generated_pictures/sos_d.pdf}}
  \end{minipage}
  \ar@{->}[d]&&&&\cr
  &
  \begin{minipage}{1.cm}\scalebox{0.35}{\includegraphics{generated_pictures/egf_r.pdf}}\end{minipage}
    \ar@{->}[dl]
    \ar@{->}[drrr]
    \ar@{->}[ddrrrr]
    \ar@{->}@/_{0.6cm}/[ddrr]
    \ar@{->}[dr]
    &&
  \begin{minipage}{1.cm}\scalebox{0.35}{\includegraphics{generated_pictures/grb2_a.pdf}}\end{minipage}
    \ar@{->}[rr]
    \ar@{->}[dlll]
    \ar@{->}[dl]
    \ar@{->}[dr]
    \ar@{->}[dd]
    \ar@{->}@/_{0.6cm}/[ddll]&&
  \begin{minipage}{1.cm}
    \scalebox{0.35}{\includegraphics{generated_pictures/shc_pi.pdf}}
    \end{minipage}
    \ar@{->}[dlllll]
    \ar@{->}[dlll]
    \ar@{->}[dl]
    \ar@{->}[dd]
    \ar@{->}[ddllll]\cr
  \begin{minipage}{1.cm}\scalebox{0.35}{\includegraphics{generated_pictures/egfr_c.pdf}}\end{minipage}
    \ar@{->}[rr]
    \ar@{->}[dr]
    \ar@{->}[drrr]
    \ar@{->}[drrrrr]
    \ar@{->}@(ul,dl)
    &&
  \begin{minipage}{1.cm}\scalebox{0.35}{\includegraphics{generated_pictures/egfr_r.pdf}}\end{minipage}
    \ar@{->}[ll]
    \ar@{->}[dl]
    \ar@{->}[dr]
    \ar@{->}[drrr]
    \ar@{->}[rr]&&
  \begin{minipage}{1.cm}\scalebox{0.35}{\includegraphics{generated_pictures/egfr_n.pdf}}\end{minipage}
    \ar@{->}[ll]
    \ar@{->}[dl]
    \ar@{->}[dlll]
    \ar@{->}[dr]
    \ar@{->}@(ur,dr)&&\cr
  &
  \begin{minipage}{1.cm}\scalebox{0.35}{\includegraphics{generated_pictures/egfr_l.pdf}}\end{minipage}&&
  \begin{minipage}{1.cm}\scalebox{0.35}{\includegraphics{generated_pictures/egfr_Y48.pdf}}\end{minipage}\ar@{->}[d]&&
  \begin{minipage}{1.cm}\scalebox{0.35}{\includegraphics{generated_pictures/egfr_Y68.pdf}}\end{minipage}\ar@{->}[d]\cr
  &&&
  \begin{minipage}{1.cm}\scalebox{0.35}{\includegraphics{generated_pictures/shc_Y7.pdf}}\ar@{->}[rr]\end{minipage}
  &&\begin{minipage}{1.cm}
  \scalebox{0.4}{\includegraphics{generated_pictures/grb2_b.pdf}}
\end{minipage}
&&\cr }
\end{minipage}}
\end{slide}

\begin{slide}{\frametitle{Detection of unbounded polymers}}

\vfill

We use Tarjan's algorithm to extract non trivial strongly connected components.

\vfill

Here there is only one:

\begin{equation*}\hspace*{1cm}\scalebox{2.5}{\xymatrix@C=0.5cm@R=0.35cm{\begin{minipage}{1cm}\scalebox{0.3}{\includegraphics{generated_pictures/egfr_c.pdf}}\end{minipage}
  \ar@{->}[rr]
  \ar@{->}@(ul,dl)
  &&
\begin{minipage}{1cm}\scalebox{0.3}{\includegraphics{generated_pictures/egfr_r.pdf}}\end{minipage}
  \ar@{->}[ll]
  \ar@{->}[rr]&&
\begin{minipage}{1cm}\scalebox{0.3}{\includegraphics{generated_pictures/egfr_n.pdf}}\end{minipage}
  \ar@{->}[ll]
  \ar@{->}@(ur,dr)&&}}\end{equation*}

\vfill

This is a false alarm and we do not know how to do better with this data-structure.

\vfill

\end{slide}

\begin{slide}{\frametitle{The graph of the sites \\ \textcolor{blue}{Example of the triangle}}}

\vfill

\hspace*{4cm}\scalebox{3}{\begin{minipage}{0.59\linewidth}\label{fig:abc:gs:gs}
\xymatrix@C=0.cm@R=0.35cm{
\begin{minipage}{0.6cm}
\scalebox{0.35}{\includegraphics{generated_pictures/a_b.pdf}}
\end{minipage}
\ar@{->}[rr]
&&
\begin{minipage}{0.6cm}
\scalebox{0.35}{\includegraphics{generated_pictures/b_c.pdf}}
\end{minipage}
\ar@{->}[ld]
&&&
\begin{minipage}{0.6cm}
\scalebox{0.35}{\includegraphics{generated_pictures/c_b.pdf}}
\end{minipage}
\ar@{->}[rd]
\cr
&
\begin{minipage}{0.6cm}
\scalebox{0.35}{\includegraphics{generated_pictures/c_a.pdf}}
\end{minipage}
\ar@{->}[lu]
&&&
\begin{minipage}{0.6cm}
\scalebox{0.35}{\includegraphics{generated_pictures/a_c.pdf}}
\end{minipage}
\ar@{->}[ru]
&&
\begin{minipage}{0.6cm}
\scalebox{0.35}{\includegraphics{generated_pictures/b_a.pdf}}
\end{minipage}
\ar@{->}[ll]
}\end{minipage}}

\vfill

The graph contains two cycles, despite that there is a finite number of
kinds of bio-molecular compounds.

\vfill
\end{slide}

\begin{slide}{\frametitle{Pros / Cons}}

\vfill

\begin{enumerate}
\item \textcolor{OliveGreen}{Pros}:
  \begin{itemize}
    \item can deal with self-bonds;
    \item can deal with conflicting sites;
    \item avoid combinatorial blow up.
  \end{itemize}
\vfill
\item \textcolor{red}{Cons}:
  \begin{itemize}
    \item cannot deal with structural invariants.
  \end{itemize}

  \vfill\mbox{}
\end{enumerate}
\end{slide}

{\superplan}

\begin{slide}{\frametitle{Repeatable patterns}}

  \vfill

  Coming back to repeatable patterns:

  \vfill

  \begin{minipage}{\linewidth}
  \begin{center}\scalebox{0.8}{\includegraphics{generated_pictures/repeatable_dimers.pdf}}
  \end{center}
  \end{minipage}

  \vfill



\end{slide}

\begin{slide}{\frametitle{Transitions between the edges}}


  \vfill

  We interpret repeatable patterns as sequences of oriented links: \bigskip



  \begin{minipage}{\linewidth}
  \begin{center}\scalebox{0.8}{\includegraphics{generated_pictures/repeatable_dimers.pdf}}
  \end{center}
  \end{minipage}

  \vfill

  \begin{enumerate}
    \item From left to right:

  \hspace*{3cm}\includegraphics{generated_pictures/egfr_egfr_r.pdf}
  \begin{minipage}{1cm}\vspace*{-2cm}${\longrightarrow}$\end{minipage} \includegraphics{generated_pictures/egfr_egfr_c.pdf}

  \vfill

    \item From right to left:

    \hspace*{3cm}\includegraphics{generated_pictures/egfr_egfr_n.pdf}
    \begin{minipage}{1cm}\vspace*{-2cm}${\longrightarrow}$\end{minipage} \includegraphics{generated_pictures/egfr_egfr_r.pdf}

  \end{enumerate}
  \end{slide}


  \begin{slide}{\frametitle{The graph of the links \\ \textcolor{blue}{Example of the triangle}}}

\vfill

  \hspace*{3cm}\scalebox{2}{\begin{minipage}{0.59\linewidth}\label{fig:abc:gl:gl}
  \xymatrix@C=0.cm@R=0.35cm{
  \begin{minipage}{1.2cm}
  \scalebox{0.35}{\includegraphics{generated_pictures/link_a_b.pdf}}
  \end{minipage}
  \ar@{->}^{\scalebox{0.2}{\includegraphics{generated_pictures/link_a_b_c.pdf}}}[rr]
  &&
  \begin{minipage}{1.2cm}
  \scalebox{0.35}{\includegraphics{generated_pictures/link_b_c.pdf}}
  \end{minipage}
  \ar@{->}^{\scalebox{0.2}{\includegraphics{generated_pictures/link_b_c_a.pdf}}}[ld]
  &&&
  \begin{minipage}{1.2cm}
  \scalebox{0.35}{\includegraphics{generated_pictures/link_c_b.pdf}}
  \end{minipage}
  \ar@{->}^{\scalebox{0.2}{\includegraphics{generated_pictures/link_c_a_b.pdf}}}[rd]
  \cr
  &
  \begin{minipage}{1.2cm}
  \scalebox{0.35}{\includegraphics{generated_pictures/link_c_a.pdf}}
  \end{minipage}
  \ar@{->}^{\scalebox{0.2}{\includegraphics{generated_pictures/link_c_b_a.pdf}}}[lu]
  &&&
  \begin{minipage}{1.2cm}
  \scalebox{0.35}{\includegraphics{generated_pictures/link_a_c.pdf}}
  \end{minipage}
  \ar@{->}^{\scalebox{0.2}{\includegraphics{generated_pictures/link_a_c_b.pdf}}}[ru]
  &&
  \begin{minipage}{1.2cm}
  \scalebox{0.35}{\includegraphics{generated_pictures/link_b_a.pdf}}
  \end{minipage}
  \ar@{->}^{\scalebox{0.2}{\includegraphics{generated_pictures/link_b_a_c.pdf}}}[ll]
  }\end{minipage}}

\vfill

  The graph contains two cycles despite that there is a bounded number
  of different kinds of bio-molecular compounds.

\vfill

  \end{slide}


\newcommand{\minipagesize}{1.1cm}
\newcommand{\factor}{0.13}

\newcommand{\graphlink}{%
\scalebox{3.}{%
  {\begin{minipage}{0.59\linewidth}\label{fig:egfr:gl:gl}
  \xymatrix@C=0.cm@R=0.35cm{
  &&&
  \begin{minipage}{\minipagesize}\scalebox{\factor}{\includegraphics{generated_pictures/sos_grb2.pdf}}\end{minipage}
  \ar@{->}[rr]
  \ar@{->}[d]&&
  \begin{minipage}{\minipagesize}\scalebox{\factor}{\includegraphics{generated_pictures/grb2_shc.pdf}}\end{minipage}
    \ar@{->}[d]
  &&\cr
  &
  \begin{minipage}{\minipagesize}\scalebox{\factor}{\includegraphics{generated_pictures/egf_egfr.pdf}}\end{minipage}
    \ar@{->}[dl]
    \ar@{->}[drrr]
    \ar@{->}[ddrrrr]
    \ar@{->}@/_{0.6cm}/[ddrr]
    \ar@{->}[dr]
    &&
  \begin{minipage}{\minipagesize}\scalebox{\factor}{\includegraphics{generated_pictures/grb2_egfr.pdf}}\end{minipage}
    \ar@{->}[dlll]
    \ar@{->}[dl]
    \ar@{->}[dr]
    \ar@{->}[dd]
    \ar@{->}@/_{0.6cm}/[ddll]&&
  \begin{minipage}{\minipagesize}\scalebox{\factor}{\includegraphics{generated_pictures/shc_egfr.pdf}}\end{minipage}
    \ar@{->}[dlllll]
    \ar@{->}[dlll]
    \ar@{->}[dl]
    \ar@{->}[dd]
    \ar@{->}[ddllll]\cr%
  \begin{minipage}{\minipagesize}\hspace*{0.2cm}\scalebox{\factor}{\includegraphics{generated_pictures/egfr_egfr_c.pdf}}
  \end{minipage}
    \ar@{->}[rr]
    \ar@{->}[dr]
    \ar@{->}[drrr]
    \ar@{->}[drrrrr]
    \ar@{->}@(ul,dl)
    &&
  \begin{minipage}{\minipagesize}\scalebox{\factor}{\includegraphics{generated_pictures/egfr_egfr_r.pdf}}\end{minipage}
    \ar@{->}[ll]
    \ar@{->}[dl]
    \ar@{->}[dr]
    \ar@{->}[drrr]
    \ar@{->}[rr]&&
  \begin{minipage}{\minipagesize}
    \scalebox{\factor}{\includegraphics{generated_pictures/egfr_egfr_n.pdf}}
  \end{minipage}
    \ar@{->}[ll]
    \ar@{->}[dl]
    \ar@{->}[dlll]
    \ar@{->}[dr]
    \ar@{->}@(ur,dr)&&\cr
  &
  \begin{minipage}{\minipagesize}\scalebox{\factor}{\includegraphics{generated_pictures/egfr_egf.pdf}}\end{minipage}
  &&
  \begin{minipage}{\minipagesize}\scalebox{\factor}{\includegraphics{generated_pictures/egfr_shc.pdf}}\end{minipage}
  \ar@{->}[d]&&
  \begin{minipage}{\minipagesize}\scalebox{\factor}{\includegraphics{generated_pictures/egfr_grb2.pdf}}\end{minipage}
  \ar@{->}[d]\cr
  &&&
  \begin{minipage}{\minipagesize}\scalebox{\factor}{\includegraphics{generated_pictures/shc_grb2.pdf}}\ar@{->}[rr]\end{minipage}
  &&
  \begin{minipage}{\minipagesize}\scalebox{\factor}{\includegraphics{generated_pictures/grb2_sos.pdf}}\end{minipage}
  &&\cr
    }
  \end{minipage}}}

}

\begin{slide}{\frametitle{The graph of the links \\\textcolor{blue}{EGF example}}}

\graphlink

\end{slide}

\begin{slide}{\frametitle{Detection of unbounded polymers}}


  \vfill

  We use Tarjan's algorithm to extract non trivial strongly connected components.

  \vfill

  Here there is only one:

  \begin{equation*}\scalebox{3}{\scalebox{1.5}{\xymatrix@C=0.2cm@R=0.35cm{\begin{minipage}{\minipagesize}\scalebox{0.15}{\includegraphics{generated_pictures/egfr_egfr_c.pdf}}\end{minipage}
    \ar@{->}[rr]
    \ar@{->}@(ul,dl)
    &&
  \begin{minipage}{\minipagesize}\scalebox{0.15}{\includegraphics{generated_pictures/egfr_egfr_r.pdf}}\end{minipage}
    \ar@{->}[ll]
    \ar@{->}[rr]&&
  \begin{minipage}{\minipagesize}\scalebox{0.15}{\includegraphics{generated_pictures/egfr_egfr_n.pdf}}\end{minipage}
    \ar@{->}[ll]
    \ar@{->}@(ur,dr)&&}}}\end{equation*}

  \vfill

  This is a false alarm and we do not know how to do better with this data-structure.

  \vfill

\end{slide}

\begin{slide}{\frametitle{Pros / Cons}}

\vfill

\begin{enumerate}
\item \textcolor{OliveGreen}{Pros}:
  \begin{itemize}
    \item can deal with self-bonds;
    \item can deal with conflicting sites;
    \item avoid combinatorial blow up.
  \end{itemize}
\vfill
\item \textcolor{red}{Cons}:
  \begin{itemize}
    \item cannot deal with structural invariants.
  \end{itemize}
\end{enumerate}

\vfill \mbox{}
\end{slide}

{\superplan}

\begin{slide}{\frametitle{The graph of the links \\\textcolor{blue}{EGF example}}}

\graphlink

\end{slide}

\begin{slide}{\frametitle{Labelled edges}}

\vfill

We label each edge with the pattern that is obtained
by gluing the source and the target graphs.

\vfill

\begin{center}
\begin{minipage}{6cm}\scalebox{0.7}{\includegraphics{generated_pictures/egfr_egfr_c.pdf}}\end{minipage}
  ${\xrightarrow{\begin{minipage}{6cm}\scalebox{0.4}{\includegraphics{generated_pictures/egfr_egfr_egfr.pdf}}%
  \end{minipage}}}$%
\begin{minipage}{6cm}\scalebox{0.7}{\includegraphics{generated_pictures/egfr_egfr_r.pdf}}\end{minipage}
\end{center}

\vfill

Then we discard the edges with a label that is unreachable.
(We use the static analysis described in

and
 )

\end{slide}


\begin{slide}{\frametitle{The graph of the links (curified) \\ \textcolor{blue}{EGF example}}}

  \scalebox{2.5}{%
  \begin{minipage}{0.59\linewidth}\label{fig:egfr:gll:gl}
  \xymatrix@C=0.cm@R=0.35cm{
  &&&
  \begin{minipage}{\minipagesize}\scalebox{\factor}{\includegraphics{generated_pictures/sos_grb2.pdf}}\end{minipage}
  \ar@{->}[rr]
  \ar@{->}[d]&&
  \begin{minipage}{\minipagesize}\scalebox{\factor}{\includegraphics{generated_pictures/grb2_shc.pdf}}\end{minipage}
    \ar@{->}[d]
  &&\cr
  &
  \begin{minipage}{\minipagesize}\scalebox{\factor}{\includegraphics{generated_pictures/egf_egfr.pdf}}\end{minipage}
    \ar@{->}[dl]
    \ar@{->}[drrr]
    \ar@{->}[ddrrrr]
    \ar@{->}@/_{0.6cm}/[ddrr]
    \ar@{->}[dr]
    &&
  \begin{minipage}{\minipagesize}\scalebox{\factor}{\includegraphics{generated_pictures/grb2_egfr.pdf}}\end{minipage}
    \ar@{->}[dlll]
    \ar@{->}[dl]
    \ar@{->}[dr]
    \ar@{->}[dd]
    \ar@{->}@/_{0.6cm}/[ddll]&&
  \begin{minipage}{\minipagesize}\scalebox{\factor}{\includegraphics{generated_pictures/shc_egfr.pdf}}\end{minipage}
    \ar@{->}[dlllll]
    \ar@{->}[dlll]
    \ar@{->}[dl]
    \ar@{->}[dd]
    \ar@{->}[ddllll]\cr%
  \begin{minipage}{\minipagesize}\hspace*{0.2cm}\scalebox{\factor}{\includegraphics{generated_pictures/egfr_egfr_c.pdf}}
  \end{minipage}
    %\ar@{->}[rr]
    \ar@{->}[dr]
    \ar@{->}[drrr]
    \ar@{->}[drrrrr]
    %\ar@{->}@(ul,dl)
    &&
  \begin{minipage}{\minipagesize}\scalebox{\factor}{\includegraphics{generated_pictures/egfr_egfr_r.pdf}}\end{minipage}
    %\ar@{->}[ll]
    \ar@{->}[dl]
    \ar@{->}[dr]
    \ar@{->}[drrr]
    %\ar@{->}[rr]
    &&
  \begin{minipage}{\minipagesize}
    \scalebox{\factor}{\includegraphics{generated_pictures/egfr_egfr_n.pdf}}
  \end{minipage}
    %\ar@{->}[ll]
    \ar@{->}[dl]
    \ar@{->}[dlll]
    \ar@{->}[dr]
    %\ar@{->}@(ur,dr)
    &&\cr
  &
  \begin{minipage}{\minipagesize}\scalebox{\factor}{\includegraphics{generated_pictures/egfr_egf.pdf}}\end{minipage}
  &&
  \begin{minipage}{\minipagesize}\scalebox{\factor}{\includegraphics{generated_pictures/egfr_shc.pdf}}\end{minipage}
  \ar@{->}[d]&&
  \begin{minipage}{\minipagesize}\scalebox{\factor}{\includegraphics{generated_pictures/egfr_grb2.pdf}}\end{minipage}
  \ar@{->}[d]\cr
  &&&
  \begin{minipage}{\minipagesize}\scalebox{\factor}{\includegraphics{generated_pictures/shc_grb2.pdf}}\ar@{->}[rr]\end{minipage}
  &&
  \begin{minipage}{\minipagesize}\scalebox{\factor}{\includegraphics{generated_pictures/grb2_sos.pdf}}\end{minipage}
  &&\cr
    }
  \end{minipage}}

\begin{center}
This graph is \textcolor{red}{acyclic}.
\end{center}

\end{slide}
\begin{slide}{\frametitle{Refinement}}

\end{slide}

{\superplan}

\end{document}
